%%%%%%%%%%%%%%%%%%%%%%%%%%%%%%%%%%%%%%%%%%%%%%%%%%%%%%%%%%%%%%%%%%%%%%%%%%%%
% AGUJournalTemplate.tex: this template file is for articles formatted with LaTeX
%
% This file includes commands and instructions
% given in the order necessary to produce a final output that will
% satisfy AGU requirements, including customized APA reference formatting.
%
% You may copy this file and give it your
% article name, and enter your text.
%
% guidelines and troubleshooting are here: 

%% To submit your paper:
\documentclass[draft]{agujournal2019}
\usepackage{url} %this package should fix any errors with URLs in refs.
\usepackage{lineno}
\usepackage[inline]{trackchanges} %for better track changes. finalnew option will compile document with changes incorporated.
\usepackage{soul}
\linenumbers
%%%%%%%
% As of 2018 we recommend use of the TrackChanges package to mark revisions.
% The trackchanges package adds five new LaTeX commands:
%
%  \note[editor]{The note}
%  \annote[editor]{Text to annotate}{The note}
%  \add[editor]{Text to add}
%  \remove[editor]{Text to remove}
%  \change[editor]{Text to remove}{Text to add}
%
% complete documentation is here: http://trackchanges.sourceforge.net/
%%%%%%%

\draftfalse
\journalname{Geophysical Research Letters}


\begin{document}

%%%%%%%%%%%%%%%%%%%%%%%%%%%%%%%%%%%%%%%%%%%%%%%
%  TITLE
%
%
%%%%%%%%%%%%%%%%%%%%%%%%%%%%%%%%%%%%%%%%%%%%%%%

\title{Constrained Circulation Ensemble}

%%%%%%%%%%%%%%%%%%%%%%%%%%%%%%%%%%%%%%%%%%%%%%%
%
%  AUTHORS AND AFFILIATIONS
%
%%%%%%%%%%%%%%%%%%%%%%%%%%%%%%%%%%%%%%%%%%%%%%%


\authors{D. Kennedy\affil{1,2}, D. M. Lawrence\affil{1}, I. R. Simpson\affil{1}}
\affiliation{1}{Climate and Global Dynamics Laboratory, NSF-NCAR}
\affiliation{2}{Earth Research Institute, University of California Santa Barbara}
\correspondingauthor{Daniel Kennedy}{djk2120@ucar.edu}



%%%%%%%%%%%%%%%%%%%%%%%%%%%%%%%%%%%%%%%%%%%%%%%
% KEY POINTS
%%%%%%%%%%%%%%%%%%%%%%%%%%%%%%%%%%%%%%%%%%%%%%%
%  List up to three key points (at least one is required)
%  Key Points summarize the main points and conclusions of the article
%  Each must be 140 characters or fewer with no special characters or punctuation and must be complete sentences

% Example:
% \begin{keypoints}
% \item	List up to three key points (at least one is required)
% \item	Key Points summarize the main points and conclusions of the article
% \item	Each must be 140 characters or fewer with no special characters or punctuation and must be complete sentences
% \end{keypoints}

\begin{keypoints}
\item enter point 1 here
\item enter point 2 here
\item enter point 3 here
\end{keypoints}

%%%%%%%%%%%%%%%%%%%%%%%%%%%%%%%%%%%%%%%%%%%%%%%
%
%  ABSTRACT and PLAIN LANGUAGE SUMMARY
%
% A good Abstract will begin with a short description of the problem
% being addressed, briefly describe the new data or analyses, then
% briefly states the main conclusion(s) and how they are supported and
% uncertainties.

% The Plain Language Summary should be written for a broad audience,
% including journalists and the science-interested public, that will not have 
% a background in your field.
%
% A Plain Language Summary is required in GRL, JGR: Planets, JGR: Biogeosciences,
% JGR: Oceans, G-Cubed, Reviews of Geophysics, and JAMES.
% see http://sharingscience.agu.org/creating-plain-language-summary/)
%
%%%%%%%%%%%%%%%%%%%%%%%%%%%%%%%%%%%%%%%%%%%%%%%

%% \begin{abstract} starts the second page

\begin{abstract}
[ enter your Abstract here ]
\end{abstract}

\section*{Plain Language Summary}
Enter your Plain Language Summary here or delete this section.
Here are instructions on writing a Plain Language Summary: 
https://www.agu.org/Share-and-Advocate/Share/Community/Plain-language-summary





%%%%%%%%%%%%%%%%%%%%%%%%%%%%%%%%%%%%%%%%%%%%%%%
%
%  INTRODUCTION
%
%%%%%%%%%%%%%%%%%%%%%%%%%%%%%%%%%%%%%%%%%%%%%%%

\section{Introduction}
The summer of 2020 was the driest of the meteorological record within the Southwest United States \cite{mankin2021,hoell2022}, capping off an extended twenty year megadrought \cite{williams2022}.
Models incorrectly predict positive vapor pressure trends in semi-arid regions \cite{simpson2024}.


%%%%%%%%%%%%%%%%%%%%%%%%%%%%%%%%%%%%%%%%%%%%%%%
%
%  METHODS
%
%%%%%%%%%%%%%%%%%%%%%%%%%%%%%%%%%%%%%%%%%%%%%%%

\section{Methods}

Our study focused on the `four corners' regions of the western United States, which is defined as the region encompassing four adjoining states: Colorado, Utah, New Mexico, and Arizona.
This region was chosen due to the spatial footprint of the 2020 drought (Supp Figure 1), as well as its significance for water management \cite{mankin2021} 
Our goal was to create a large sample of simulations exhibiting extreme drought conditions, while still allowing for local land-atmosphere coupling, in order to investigate the influences of antecedent conditions and climate change.

We ran multiple ensembles of simulations using CESM2, each consisting of 30 members, with varied initial conditions. For each simulation, winds were nudged to the ERA5 reanalysis for the year 2020. 
Nudging occurred four times per day in each grid cell outside the box bounded by 28$^{\circ}$N, 47$^{\circ}$N, 130$^{\circ}$W, 96$^{\circ}$W. This serves to constrain the large-scale circulation, imposing meteorological drought conditions for every ensemble member. Inside the bounding box, the atmosphere evolves freely, allowing for full land-atmosphere coupling, which is necessary for studying the effects of initial conditions.  
Sea surface temperatures (SSTs) were imposed from an ERA5 reanalysis product, and all other forcing (e.g. CO$_2$, aerosols) followed the standard conventions for transient simulations as in the CESM2-LE. 
The simulations were initialized in April, using random initial conditions from the CESM2-LE for all the requisite land and atmosphere states (e.g. temperature, soil moisture, humidity). 

Our primary ensemble was based in 2020, mirroring the extreme 2020 Southwest U.S. drought. We also simulated this drought during pre-industrial and future conditions to investigate the influence of climate change. All of the ensembles were nudged to the 2020 winds, but initial conditions and other climate forcings were drawn from the appropriate time period mirroring the CESM2-LE, effectively porting the 2020 circulation to these two alternative climates. SSTs were anomaly-based, computing anomalies from the CESM2-LE and applying them to the ERA5 reanalysis SSTs used in the 2020 ensemble. Initial conditions were drawn at random from the CESM2-LE from either April 1850 or April 2090, and all other forcing (e.g. CO$_2$, aerosols) followed the standard conventions for transient simulations as in the CESM2-LE, which follows the SSP3-7.0 emissions scenario.

We have compared our ensembles to the existing CESM2-LE \cite{rodgers2021}, which consists of 100 transient simulations (1850-2100) of the fully coupled configuration of CESM2.
This ensemble is useful, because it provides a large number of model realizations, sampling across a range of internal climate variability.
For the sake of comparison, we have utilized the same model source code and utilized the same forcing where applicable.
The two key differences between our ensembles and the CESM2-LE are that 1) we impose the 2020 drought circulation via nudging and 2) we forced the simulations with observed SSTs (in lieu of dynamically coupled SSTs). 

 \begin{table}[h]
 \caption{Ensemble descriptions}
 \centering
 \begin{tabular}{l c c c c}
 \hline
  Name  & Winds & Climate forcing & SSTs & Initial Conditions \\
 \hline
   Control  & ERA5-2020& 2020 (SSP3-7.0) & ERA5-2020 & CESM2-LE (2020)\\
   Pre-industrial  & ERA5-2020 & 1850 & ERA5-2020 + CESM2-LE anomalies (1850) & CESM2-LE (1850) \\
   Future        & ERA5-2020 & 2090 (SSP3-7.0) & ERA5-2020 + CESM2-LE anomalies (2020) & CESM2-LE (2090) \\
 \hline
 \end{tabular}
 \label{tab:exps}
 \end{table}








%%%%%%%%%%%%%%%%%%%%%%%%%%%%%%%%%%%%%%%%%%%%%%%
%
%  OUTLINE
%
%%%%%%%%%%%%%%%%%%%%%%%%%%%%%%%%%%%%%%%%%%%%%%%

\section{Outline and in flux}

\subsection{Key points}
\begin{itemize}
	\item CCE can reproduce the 2020 drought
	\item Soil water initialization influences drought severity via variables (soil moisture, runoff, temperature extremes), but does not exacerbate the meteorological drought, itself 
	\item Warming affects this drought
	\begin{itemize}
		\item ET reduced in both free-running and constrained droughts
		\item P reduced in CCE, but mixed signal in free-running CESM2-LE
	\end{itemize}
	\item 2090 CCE is much drier than the CESM2-LE
\end{itemize}


\subsection{Some numbers I'd like}
\begin{itemize}
\item 2020 JAS Precip (obs or CCE?) sits at the XXth percentile of the CESM2 large ensemble. 
\end{itemize}

\subsection{Supp figures I'll need}
\begin{itemize}
\item CESM2-LE JAS Precip historgram with obs and/or CCE
\end{itemize}


%%%%%%%%%%%%%%%%%%%%%%%%%%%%%%%%%%%%%%%%%%%%%%%
%
%  DISCUSSION
%
%%%%%%%%%%%%%%%%%%%%%%%%%%%%%%%%%%%%%%%%%%%%%%%

\section{Results and Discussion}

\begin{figure}
\includegraphics[width=32pc]{../figs/main/domain.pdf}
\caption{Map of the study domain and nudging boundary. All analyses were based in the `four corners' region of the western United States, yellow area. Model simulations were nudged to reanalysis winds outside of the red box, in order to induce the observed large-scale circulation, while allowing for local land-atmosphere coupling.}
\end{figure}

\begin{figure}
\includegraphics[width=28pc]{../figs/main/precip.pdf}
\caption{Precipitation climatologies for the four corners region from reanalysis (GPCP, 1981-2020) and a model large ensemble (CESM2-LE, 1981-2020) alongside our constrained circulation ensemble (CCE-2020). For each of the datasets, shading spans the 5th to 95th percentiles (across years and/or ensemble members), and the solid line tracks the mean. The dotted black line shows the reanalysis precipitation for 2020. The CESM2-LE shows a small high bias in mean and variance relative to reanalysis precipitation, but does not effectively capture the timing or magnitude of the summer monsoon. When nudged to 2020 winds, CESM2 can reproduce the 2020 drought, even though it falls outside the large ensemble envelope.
N.b. will combine this with Figure 1.}
\end{figure}

\begin{figure}
\includegraphics[width=35pc]{../figs/main/inits.pdf}
\caption{The influence of soil water initialization on various summer (JAS) drought indicators in the CCE, shown in the context of the variance across the CESM2-LE. In each case, the 10cm soil water content at initialization is plotted (red dots) against the summer drought indicator, with linear regression where appropriate (red lines). The colored stripes span the CESM2-LE deciles for the given drought indicator, truncated at the 1st and 99th percentiles. Dots in the white background area, exist below the 1st percentile of the CESM2-LE. Initial soil water content influences many drought indicators, but not the severity of the meteorological drought itself, i.e. summer precipitation.}
\end{figure}

\begin{figure}
\includegraphics[width=35pc]{../figs/main/scatter_ET_P.png}
\caption{Summer evapotranspiration vs. precipitation in the CCE (a) and a dry subset of the CESM2-LE (d), alongside histograms of ET (b,e) and precipitation (c,f) across three time periods. The CESM2-LE is subset to the 30 driest ensemble-years for each period from 1000 total ensemble-years (100 members, 10 years around the given year) based on JAS 10cm soil water content. Stars indicate significant differences in the mean relative to the two other time periods, via t-test at the p$<$0.05 level. Forced responses tend to reduce ET and precipitation in both the CCE and the dry subset of the CESM2-LE. }
\end{figure}


\begin{figure}
\includegraphics[width=35pc]{../figs/main/contours_ET_SW.pdf}
\caption{Summer evapotranspiration vs. 10cm soil water content from the CCE (dots) and the CESM2-LE (scatter density contours) across the three time periods. The relationship in the CCE closely follows the CESM2-LE pattern. The CCE exists at the dry extreme of the CESM2-LE, or even beyond, in the 2090 ensemble. }
\end{figure}


\begin{figure}
\includegraphics[width=35pc]{../figs/main/contours_PREC_ET.pdf}
\caption{Summer precipitation vs. evapotranspiration from the CCE (dots) and the CESM2-LE (scatter density contours) across the three time periods. The location of the CCE data comports with the CESM2-LE pattern, but the slopes do not appear to match. N.b. will combine this with Figure 5.}
\end{figure}



%%%%%%%%%%%%%%%%%%%%%%%%%%%%%%%%%%%%%%%%%%%%%%%
%
%  CONCLUSION
%
%%%%%%%%%%%%%%%%%%%%%%%%%%%%%%%%%%%%%%%%%%%%%%%

\section{Conclusions}
%Text here ===>>>





%%%%%%%%%%%%%%%%%%%%%%%%%%%%%%%%%%%%%%%%%%%%%%%
%
% DATA SECTION and ACKNOWLEDGMENTS
%
%%%%%%%%%%%%%%%%%%%%%%%%%%%%%%%%%%%%%%%%%%%%%%%

\section*{Open Research Section}
This section MUST contain a statement that describes where the data supporting the conclusions can be obtained. Data cannot be listed as ''Available from authors'' or stored solely in supporting information. Citations to archived data should be included in your reference list. Wiley will publish it as a separate section on the paper’s page. Examples and complete information are here:
https://www.agu.org/Publish with AGU/Publish/Author Resources/Data for Authors




\acknowledgments
Enter acknowledgments here. This section is to acknowledge funding, thank colleagues, enter any secondary affiliations, and so on.


%%%%%%%%%%%%%%%%%%%%%%%%%%%%%%%%%%%%%%%%%%%%%%%
%
% REFERENCES and BIBLIOGRAPHY
%
%%%%%%%%%%%%%%%%%%%%%%%%%%%%%%%%%%%%%%%%%%%%%%%

\bibliography{ refs.bib }




\end{document}





