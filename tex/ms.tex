\documentclass[11pt]{article}
\usepackage{amsmath,amssymb}
\usepackage[makeroom]{cancel}
\usepackage{graphicx}
\usepackage{array}
\usepackage{changes}
\usepackage{natbib}
\graphicspath{ ../figs/}
\usepackage[margin=1in]{geometry}
\usepackage[parfill]{parskip}  
\title{Southwest US drought dynamics \large \\}
\author{Daniel Kennedy - djk2120@ucar.edu \\ Isla Simpson }

\begin{document}
\maketitle

\section{Introduction}

\begin{itemize}
    \item SWUS experienced a megadrought, and 2020 was the driest year on record.
    \item The 2020 drought was associated with xyz
    \item Climate change is not understood to have greatly exacerbated the 2020 drought.
    \item Whether average precipitation is projected to increase, decrease, or remain the same varies among climate models
    \item CESM2 projects that average precipitation will stay the same
    \item The statistics of determining whether extreme events are getting more extreme is more tenuous given the smaller sample size.
\end{itemize}

\newpage
Key points:
\begin{itemize}
\item The 2020 SWUS monsoon precipitation was extremely low, but within the envelope of the CESM2-LE (exact percentile?)
\item The CCE can reproduce the 2020 SWUS drought (precip), with severe drought in all 30 simulations.
\item Porting the 2020 circulation to 1850 and 2090 climates, yields significantly reduced precip with warming -6\% (1850$\rightarrow$2020) and -18\% (1850$\rightarrow$2090).
\item Though the fully coupled model does not have a consistent precip decline in response to warming on average, precip does decline in dry subsets of the ensemble.
\item The 2090 CCE simulations are more extreme relative to the large ensemble than the 1850 or 2020 simulations (SM, ET, T, P, VPD).
\end{itemize}



\begin{figure}[h]
\centering
\includegraphics[width=30pc]{figs/domain.pdf}
\caption{Map of our study domain (yellow) and the nudging box (red).}
\label{fig:domain}
\end{figure}


\newpage

\begin{figure}[h]
\centering
\includegraphics[width=25pc]{figs/main/precip.pdf}
\caption{Precip in the four corners.}
\label{fig:precip}
\end{figure}


\newpage




\newpage
\begin{figure}[h]
\centering
\includegraphics[width=40pc]{figs/main/scatter_ET_P.png}
\caption{Precipitation vs. ET in the constrained ensemble and the driest 3\% of the large ensemble.}
\label{fig:precip}
\end{figure}








\begin{figure}[h]
\centering
\includegraphics[width=40pc]{figs/contours/SOILWATER_10CM_ET_contours.png}
\caption{[MAIN] The CCE ET$\sim$SW relationship seems consistent with the large ensemble.}
\label{fig:precip}
\end{figure}

\begin{figure}[h]
\centering
\includegraphics[width=40pc]{figs/main/anomalies.png}
\caption{Foco anomalies w/r/t global TREFHT warming.}
\label{fig:anomalies}
\end{figure}


\clearpage

supp figs
\begin{figure}[h]
\centering
\includegraphics[width=40pc]{figs/contours/ET_PREC_contours.png}
\caption{[SUPP] The CCE P$\sim$ET seems somewhat different.}
\label{fig:precip}
\end{figure}

\begin{figure}[h]
\centering
\includegraphics[width=40pc]{figs/contours/TREFHTMX_TSA_contours.png}
\caption{[SUPP] The CCE is not very hot, generally. But the daily max temperature increases quite a lot in the 2090 case. }
\label{fig:precip}
\end{figure}

\begin{figure}[h]
\centering
\includegraphics[width=40pc]{figs/contours/TREFHTMX_VPD_contours.png}
\caption{[SUPP] The CCE air is very dry, especially in 2090. Partly due to a stronger warming effect in the CCE than in the coupled model.}
\label{fig:precip}
\end{figure}


\begin{figure}[h]
\centering
\includegraphics[width=40pc]{figs/contours/PRECC_PRECL_contours.png}
\caption{[SUPP] The CCE is extreme in convective precipitation as opposed to large-scale? Both are decreasing with warming.}
\label{fig:precip}
\end{figure}

\clearpage
SWUS JAS Rain and ET in these ensembles:
\begin{itemize}
    \item In the CCE: precip and ET decrease with warming.
    \item In CESM2-LE on average, precip is highest in 1850, lowest in 2020, and in between those two in 2090.
    \item In CESM2-LE dry composites, 1850 has more rain, 2020 and 2090 are not distinguishable
    \item In CESM2-LE on average, ET increases with warming.
    \item In CESM2-LE, evaporation \textbf{decreases} with warming in all of the dry composites    
\end{itemize}

\bibliographystyle{abbrvnat}
\nocite{*}
\bibliography{refs/all}
\end{document}




