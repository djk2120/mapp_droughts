\documentclass[11pt]{article}
\usepackage{amsmath,amssymb}
\usepackage[makeroom]{cancel}
\usepackage{graphicx}
\usepackage{array}
\usepackage{changes}
\usepackage{natbib}
\graphicspath{ ../figs/}
\usepackage[margin=1in]{geometry}
\usepackage[parfill]{parskip}  
\title{Southwest US drought dynamics \large \\}
\author{Daniel Kennedy - djk2120@ucar.edu \\ Isla Simpson }

\begin{document}
\maketitle

\section{Introduction}

\begin{figure}[h]
\centering
\includegraphics[width=40pc]{figs/domain.pdf}
\caption{Map of our study domain (yellow) and the nudging box (red).}
\label{fig:domain}
\end{figure}


\begin{itemize}
    \item SWUS experienced a megadrought, and 2020 was the driest year on record.
    \item The 2020 drought was associated with xyz
    \item The extent to which climate change may have exacerbated the 2020 drought is still being debated.
    \item whether average precipitation is projected to increase, decrease, or remain the same varies among climate models
    \item CESM2 projects that average precipitation will stay the same
    \item The statistics of determining whether the droughts are getting drier is more tenuous given the smaller sample size.
\end{itemize}


Drought in the large ensemble:
\begin{itemize}
    \item CESM2 does not have a very strong NAM, and its timing is late, peaking in September.
    \item September also features the largest precipitation variance.
    \item September precipitation tends to exceed evapotranspiration, implying moisture convergence to the region. 
    \item Soil moisture initialization is not a strong control on JAS precip.
    
    
\end{itemize}


\bibliographystyle{abbrvnat}
\nocite{*}
\bibliography{refs/all}
\end{document}




